
% please use TexLive 2014 or later with the M&C macros freely
% available from tug.org or use any other recent version of LaTeX

\documentclass{article}\usepackage[]{graphicx}\usepackage[table]{xcolor}
% maxwidth is the original width if it is less than linewidth
% otherwise use linewidth (to make sure the graphics do not exceed the margin)
\makeatletter
\def\maxwidth{ %
  \ifdim\Gin@nat@width>\linewidth
    \linewidth
  \else
    \Gin@nat@width
  \fi
}
\makeatother

\definecolor{fgcolor}{rgb}{0.345, 0.345, 0.345}
\newcommand{\hlnum}[1]{\textcolor[rgb]{0.686,0.059,0.569}{#1}}%
\newcommand{\hlstr}[1]{\textcolor[rgb]{0.192,0.494,0.8}{#1}}%
\newcommand{\hlcom}[1]{\textcolor[rgb]{0.678,0.584,0.686}{\textit{#1}}}%
\newcommand{\hlopt}[1]{\textcolor[rgb]{0,0,0}{#1}}%
\newcommand{\hlstd}[1]{\textcolor[rgb]{0.345,0.345,0.345}{#1}}%
\newcommand{\hlkwa}[1]{\textcolor[rgb]{0.161,0.373,0.58}{\textbf{#1}}}%
\newcommand{\hlkwb}[1]{\textcolor[rgb]{0.69,0.353,0.396}{#1}}%
\newcommand{\hlkwc}[1]{\textcolor[rgb]{0.333,0.667,0.333}{#1}}%
\newcommand{\hlkwd}[1]{\textcolor[rgb]{0.737,0.353,0.396}{\textbf{#1}}}%
\let\hlipl\hlkwb

\usepackage{framed}
\makeatletter
\newenvironment{kframe}{%
 \def\at@end@of@kframe{}%
 \ifinner\ifhmode%
  \def\at@end@of@kframe{\end{minipage}}%
  \begin{minipage}{\columnwidth}%
 \fi\fi%
 \def\FrameCommand##1{\hskip\@totalleftmargin \hskip-\fboxsep
 \colorbox{shadecolor}{##1}\hskip-\fboxsep
     % There is no \\@totalrightmargin, so:
     \hskip-\linewidth \hskip-\@totalleftmargin \hskip\columnwidth}%
 \MakeFramed {\advance\hsize-\width
   \@totalleftmargin\z@ \linewidth\hsize
   \@setminipage}}%
 {\par\unskip\endMakeFramed%
 \at@end@of@kframe}
\makeatother

\definecolor{shadecolor}{rgb}{.97, .97, .97}
\definecolor{messagecolor}{rgb}{0, 0, 0}
\definecolor{warningcolor}{rgb}{1, 0, 1}
\definecolor{errorcolor}{rgb}{1, 0, 0}
\newenvironment{knitrout}{}{} % an empty environment to be redefined in TeX

\usepackage{alltt}

%the main style; default LibreCaslon font
% \usepackage[raggedsec,sectionbib]{morgan2}
% \usepackage{morgan-defs}

%to use Times New Roman, instead of LibreCaslon, please uncomment the next line
%\morgansetup{fontsetup=times}

% bibliography
% \usepackage[comma,sort,authoryear]{natbib}         % author-year
% \usepackage[square,comma,sort&compress,numbers]{natbib} % numbered
% \usepackage{makeidx}     % Per Tondo's suggestion on 1 Feb 2022.
% \usepackage{chapterbib}
% 
% % 
% % Begin LaTeX packages imported by the authors.
% % 
% 
% \usepackage{booktabs}    % For awesome table formatting
% \usepackage{caption}     % Necessary for the subcaption package?
% \usepackage{epigraph}    % For epigraphs below chapter titles.
% \usepackage{eurosym}     % For the EU currency symbol.
% \usepackage{gensymb}     % For the \degree command.
% \usepackage{microtype}   % For (more) beautiful typesetting.
% \usepackage{rotating}    % For rotating figures.
% \usepackage[normalem]{ulem} % For strikethrough text.
% \usepackage{subcaption}  % For labeling subfigures.
% \usepackage{tikz}
% \usetikzlibrary{arrows}
% \usetikzlibrary{positioning}
% \usetikzlibrary{shapes.geometric}
% \usepackage{wrapfig}     % To wrap text around figures.
% \usepackage[table]{xcolor}      % makes colored text.
% 
% % The morgan-defs.sty file already loads the enumitem package.
% % Add the inline option here
% % for inline enumerate* lists.
% \PassOptionsToPackage{inline}{enumitem}
% 
% % 
% % Begin LaTeX macros created by the authors.
% % 
% 
% % Temperatures
% \newcommand{\degC}{\degree C}
% \newcommand{\degF}{\degree F}
% % Figure widths
% \newcommand{\figwidth}{0.85\linewidth}
% % The "math font trick" to get Libre Caslon font in italics inside equations.
% \newcommand{\mft}[1]{\text{\emph{#1}}}
% \newcommand{\ICOtwonodollar}{\mft{I}_{\text{CO}_2}}
% \newcommand{\ICOtwo}{$\ICOtwonodollar{}$}
% \newcommand{\ICOtwoparen}{(\ICOtwo{})}
% \newcommand{\Iparen}{(\:\!\mft{I}\:\!)}
% \newcommand{\Pparen}{(\:\!\mft{P}\;\!)}
% \newcommand{\Aparen}{(\;\!\mft{A}\:\!)}
% \newcommand{\Rparen}{(\:\!\mft{R}\;\!)}
% \newcommand{\REpnodollar}{\mft{R}_{\mft{Ep}}}
% \newcommand{\REp}{$\REpnodollar$}
% \newcommand{\REpparen}{(\:\!\REp)}
% \newcommand{\Xparen}{(\;\!\mft{X}\:)}
% \newcommand{\XCOtwonodollar}{\mft{X}_{\text{CO}_2}}
% \newcommand{\XCOtwo}{$\XCOtwonodollar$}
% \newcommand{\XCOtwoparen}{(\;\!\XCOtwo)}
% \newcommand{\Tparen}{(\mft{T}\:)}
% 
% \newcommand{\rateICOtwo}{$\mft{r}_{\ICOtwonodollar{}}$}
% \newcommand{\rateP}{$\mft{r}_{\mft{P}}$}
% \newcommand{\rateA}{$\mft{r}_{\mft{A}}$}
% \newcommand{\rateREp}{$\mft{r}_{\REpnodollar{}}$}
% \newcommand{\rateXCOtwo}{$\mft{r}_{\XCOtwonodollar{}}$}
% 
% 
% % IPARX equation
% \newcommand{\iparx}{
% \footnotesize
% \begin{tabular}{ccccccccc}
% \mft{I}  & = & \mft{P}        & $\times$ & \mft{A}       & $\times$ & \mft{R}      & $\times$ & \mft{X} \\
% \multirow{2}{*}{Impact} &  & \multirow{2}{*}{Population} &  & \multirow{2}{*}{Affluence} &  & Resource intensity &  & Impact \\
%  &  &  &  &  &  & of the economy &  & of resources \\
% $\left[\frac{\text{impact}}{\text{year}}\right]$ & & $\left[\text{persons}\right]$ &  & $\left[\frac{\text{\$ GDP}}{\text{person$\cdot$year}}\right]$ &  & $\left[\frac{\text{resources}}{\text{\$ GDP}}\right]$ &  & $\left[\frac{\text{impact}}{\text{resources}}\right]$
% \end{tabular}
% \normalsize
% }
% 

%
% Create environments for questions and projects, 
% as suggested by Tondo on 14 July 2021.
%

% \newenvironment{questions}%
% {\begin{enumerate}
% \renewcommand{\labelenumi}{Q\thechapter.\arabic{enumi}}
% \renewcommand{\theenumi}{Q\thechapter.\arabic{enumi}}}
% {\end{enumerate}}
% 
% \newenvironment{projects}%
% {\begin{enumerate}
% \renewcommand{\labelenumi}{P\thechapter.\arabic{enumi}}
% \renewcommand{\theenumi}{P\thechapter.\arabic{enumi}}}
% {\end{enumerate}}
% 
% 
% %
% % End author additions
% % 
% 
% 
% \setcounter{secnumdepth}{2}
% 
% \graphicspath{{./figures/}}     % folder for the figures in your book
% 
% \PassOptionsToPackage{hyphens}{url}
% \usepackage[colorlinks=true,linkcolor=MyDarkBlue,
% citecolor=MyDarkBlue,filecolor=MyDarkBlue,urlcolor=MyDarkBlue]{hyperref}
% \usepackage{doi}
% 
% \renewcommand{\UrlBreaks}{\do\.\do\@\do\\\do\/\do\!\do\_\do\|\do\;\do\>\do\]%
% \do\)\do\,\do\?\do\&\do\'\do+\do\=\do\#%
% \do\a\do\b\do\c\do\d\do\e\do\f\do\g\do\h\do\i\do\j%
% \do\k\do\l\do\m\do\n\do\o\do\p\do\q\do\r\do\s\do\t%
% \do\u\do\v\do\w\do\x\do\y\do\z\do\A\do\B\do\C\do\D%
% \do\E\do\F\do\G\do\H\do\I\do\J\do\K\do\L\do\M\do\N%
% \do\O\do\P\do\Q\do\R\do\S\do\T\do\U\do\V\do\W\do\X%
% \do\Y\do\Z}
% \makeatletter
% \g@addto@macro{\UrlBreaks}{\UrlOrds}
% \renewcommand{\ALG@name}{\color{black}Algorithm}
% \makeatother



%\makeindex{}                     % if you are creating an index for your book
\IfFileExists{upquote.sty}{\usepackage{upquote}}{}
\begin{document}

\title{A Calvin Core 100 Introduction to Sustainability}
\author{Jeremy Van~Antwerp, Matthew Kuperus Heun}
%\affiliation{Engineering Department, Calvin University}
\maketitle
 
Let’s begin with a story.
% Once upon a time, a king was challenged to a game of chess by a poor scholar.
% If he lost, the king would have to pay the scholar 
% a chessboard’s worth of grain, 
% defined as one kernel of grain on the first square, 
% two grains on the second square, four grains on the third square, 
% and so on, with each subsequent square receiving 
% double what was on the previous square.
% The king accepted the wager but lost the game.
% A chessboard has eight rows and eight columns, so the king had to pay 2$^\text{64}-$1 grains.
% (See Section \ref{sec:exp} for an explanation of the math behind this formula.)
% Rice is 7,000 grains per pound ($\sim$~65~mg each),
% so the king owed the scholar 1.199~$\times$~10$^{\text{15}}$ kg of rice,
% which is more than 150 tons of rice for every person currently on planet Earth
% (worth about \$40,000 per person).
% For reference, people eat about 35 tons of food in a lifetime,
% so this is~4.4 lifetimes' supply of food for every person on the planet.
% Clearly, the scholar's winnings won't all fit on an ordinary chessboard!
% 
% This story, which goes back at least to the 1200s AD, 
% illustrates several aspects of the
% sustainability problem that have been evident for centuries:
% Exponential growth (Section \ref{sec:exp}) is
% inherently not sustainable;
% big numbers are hard to \href{https://en.wiktionary.org/wiki/grok}{grok};
% and making large wagers on limited understanding is unwise.
% At the moment, human society is making the king's wager
% writ large.
% We are pursuing exponential growth without fully understanding the consequences.
% In short, we have \textbf{sustainability challenges}.
% 
% You are reading this book because you think sustainability is important.
% And it is.
% In the long run, sustainability is one of the few things that matter at all.
% Sustainability is a challenging problem, in part due to exponential growth,
% in part due to limited understanding,
% but also because sustainability problems are complex and interconnected.
% Each of the subproblems or subdomains of sustainability may have its own problems
% of exponential growth and limited knowledge.
% 
% Furthermore, the king's wager hints at a social element of the sustainability problem.
% In the story, the king doesn't want to appear ``weak'' before his subject and so
% (rashly) accepts the wager.
% In sustainability, there is tremendous social (and sometimes economic) pressure
% to continue to do things ``the way they've always been done.''
% 
% Figure~\ref{fig:venn_diagram} shows the three interrelated and overlapping domains
% of sustainability:
% \textbf{environmental sustainability}\index{environmental sustainability},
% \textbf{economic sustainability}\index{economic sustainability}, and
% \textbf{social sustainability}\index{social sustainability}.
% Sustainability problems are complex and interconnected.
% Environmental sustainability problems have social and economic aspects (that are
% generally more difficult to solve).
% Likewise, economic and social sustainability problems are not limited to one domain;
% typically, they include aspects of the other two domains as well.
% Therefore (for example), pollution is an environmental problem, and a social problem, and an economic problem.
% Likewise renewable energy. 
% And land use. 
% And all the other challenges humanity faces.
% 
% \begin{figure}
% \centering
% 
%   % The next command tells RStudio to do "Compile PDF" on book.Rnw,
% instead of this chapter, thereby eliminating the need to switch back to book.Rnw 
% before making the book.
%!TEX root = ../../book.Rnw

\begin{tikzpicture}
    \definecolor{my_purple}{HTML}{440154}
    \definecolor{my_yellow}{HTML}{FDE725}
    \definecolor{my_green}{HTML}{21908C}
    \tikzstyle{every node}=[ultra thick, ellipse, 
                            minimum width = 150pt,
                            minimum height = 130pt]
    \node[ellipse, draw, fill = my_purple, opacity = 0.8, align = left, text = white] (social) at (-2,0) {Social};
    \node[ellipse, draw, fill = my_yellow, opacity = 0.8, align = center] (economic) at (0,3) {Economic};
    \node[ellipse, draw, fill = my_green, opacity = 0.8, align = right] (environmental) at (2,0) {Environmental};
\end{tikzpicture}

% 
%   \caption[Three aspects of sustainability]
%           {The three aspects of sustainability.
%            Sustainability is often visualized as three overlapping ellipses:
%            economic sustainability\index{economic sustainability},
%            environmental sustainability\index{environmental sustainability}, and
%            social sustainability\index{social sustainability}.
%            The intersecting area represents fully sustainable living.
%            Environmental sustainability\index{environmental sustainability}
%            refers to the \textbf{ecosystem}\index{ecosystem} and its supporting services;
%            see Chapter~\ref{chap:planetary_boundaries}.
%            Economic sustainability\index{economic sustainability}
%            refers to human systems for creating and accounting for wealth\index{wealth};
%            see Chapter~\ref{chap:affluence}.
%            Social sustainability\index{social sustainability}
%            refers to traditions and systems of human society.
%            The layering in this figure represents coverage in this book.
%            We focus on environmental\index{environmental sustainability} and
%            economic sustainability\index{economic sustainability}.
%            Social sustainability is vitally important, so it forms the backdrop
%            for much of the discussion herein.}
% \label{fig:venn_diagram}
% \end{figure}
% 
% 
% Sustainability is challenging for two additional reasons.
% First, it may seem that the changes required to achieve sustainability are so
% massively overwhelming that they shouldn't even be addressed.
% Second, it may seem like the consequences of not becoming sustainable are so far
% in the future that there is neither urgency nor immediate payback.
% However, neither of these views is true.
% Current news headlines indicate that we are already
% experiencing the effects of not being sustainable, in all
% sustainability domains---environmentally, economically, and socially.
% There are many reasonable and simple things that can be done to improve our
% sustainability in the near term.
% The deep changes needed for long-term sustainability are urgently needed precisely because
% they are long-term investments.
% The sooner we begin making those investments, the sooner we'll begin reaping the
% rewards and the greater those rewards will be.
% 
% 
% %%%%%%%%%%%%%%%%%%%%%%%%%%%%%%%%%%%%%%%%%%%%%%%%%%%%%%%%%%%%%%
% \section{Purpose and focus}
% %%%%%%%%%%%%%%%%%%%%%%%%%%%%%%%%%%%%%%%%%%%%%%%%%%%%%%%%%%%%%%
% 
% This book summarizes ways that humans are not living sustainably and suggests
% characteristics of sustainable societies.
% The text is deliberately short because the book is not intended to be comprehensive.
% The focus is on equipping the reader to discuss moral and ethical issues around sustainability.
% Because the choices and paths to sustainability are filled with value judgments and moral choices,
% the end-of-chapter discussion questions mostly point to tradeoffs and do not have ``right answers.''
% Instead, different answers indicate different preferences.
% This book presents a coherent framework for discussing sustainability
% that is grounded in a sense of scale.
% Therefore, we prioritize presenting information graphically.
% The intent is to equip readers with basic knowledge (informed by scale) so we can
% grapple with tough moral questions.
% \emph{This book should be easy to read but hard to digest.}
% As you read this book, it should raise many questions for you.
% 
% The focus of this book is the many challenges of sustainability\index{sustainability challenges}.
% While we may, at times, point to directions for improved sustainability
% (Chapters~\ref{chap:government} and~\ref{chap:personal_action}),
% it is beyond our scope
% (and indeed our ability) to provide solutions for all sustainability problems.
% Some (or many) of the questions that are raised about sustainability
% will remain unanswered.
% 
% Our framework for describing sustainability challenges\index{sustainability challenges}
% is illustrated by data.
% Thus, readers can expect graphs, tables, and other numerical representations
% of the state of our world as it relates to sustainability.
% Part~\ref{part:IPARX} (Chapters~\ref{chap:introduction}--\ref{chap:impact_intensity})
% shows mostly data from the world in aggregate, thereby eliminating issues of
% imports and exports between countries.
% Part~\ref{part:sustainability_challenges} shows data mostly from the U.S.%
% \footnote{
%   Readers are encouraged to remember that the U.S.\ is atypical in many ways.
%   Sociologically, the U.S.\ is WIERD\index{Western, industrialized, educated, rich, and democratic (WIERD)}
%   (Western, industrialized, educated, rich, and democratic).
%   Additionally, the U.S.\ has a low population density\index{population density} and lots of resources.
%   While our framework for sustainability thinking is universal, the data used for
%   illustration may not always apply to the rest of the world.
% },
% because it is our home country and
% because it has better data coverage than most countries
% regarding energy and carbon emissions\index{CO$_2$ emissions}.
% Behind the facts and data are important concepts related to human choices,
% which we explore throughout the book, mostly in questions and projects that follow each chapter.
% 
% The remainder of this chapter summarizes key themes of the book.
% 
% 
% %%%%%%%%%%%%%%%%%%%%%%%%%%%%%%%%%%%%%%%%%%%%%%%%%%%%%%%%%%%%%%
% \section{What is sustainability?}
% \label{sec:what_is_sustainability}
% %%%%%%%%%%%%%%%%%%%%%%%%%%%%%%%%%%%%%%%%%%%%%%%%%%%%%%%%%%%%%%
% 
% ``Sustainability'' is a crucial concept.
% If humanity is not living sustainably, we will either cease to exist as a species
% or (at least) experience drastic reductions in our population and/or standard of living.
% So, what is sustainability and how do we tell if we're living sustainably?
% There are many definitions of sustainability\index{sustainability!definition}.
% To the novice, many definitions may seem to indicate lack of agreement.
% However, sustainability is almost a self-defining concept.
% Different answers indicate different assumptions, different priorities, and different
% boundaries (that is, what system is being considered).
% 
% When considering the meaning of sustainability, the two most important questions are
% ``sustaining what?'' and ``for how long?''
% The second of these questions is perhaps easier to answer.
% Although humans seem to have inherent cognitive difficulties in planning for
% time scales significantly longer than the human lifespan,
% \textbf{strong sustainability}\index{strong sustainability} is achieved only if the answer is
% ``indefinitely'' or ``forever.''
% Our current sustainability crisis has been thousands of years in the making~\cite{Sanderman2017}
% and will likely require millions of years to recover~\cite{Davis11262}.
% A \textbf{weak sustainability}\index{weak sustainability} criterion is sustainability over a long (by human standards)
% time frame, perhaps 50 years, which is on the order of one human adulthood.%
% \footnote{
%   Our criteria for strong\index{strong sustainability} and weak sustainability\index{weak sustainability} are physical and
%   follow \citet{Graedel:2010ab}.
%   Other definitions for weak\index{weak sustainability} and strong sustainability\index{strong sustainability} exist.
%   One approach to weak\index{weak sustainability} and strong\index{strong sustainability} sustainability is economic.
%   It defines strong sustainability\index{strong sustainability} as protecting natural capital\index{natural capital}
%   for a certain period of time
%   (equivalent to our weak sustainability\index{weak sustainability} definition 
%   if the time horizon for protecting natural capital\index{natural capital} is 50 years).
%   In the economic approach,
%   weak sustainability\index{weak sustainability} allows 
%   produced capital\index{produced capital}
%   to substitute for natural capital\index{natural capital}
%   across a period of time.
%   See \citet{Dietz:2007tu} for details.
% }
% (Note that corporate time scales for decision making are even shorter,
% often in the 5--20 year range; see Chapter~\ref{chap:government}.)
% 
% The narrowest answer for what needs sustaining is human life and society, which
% \href{https://www.merriam-webster.com/dictionary/perforce}{perforce}
% entails those \textbf{ecosystem services}\index{ecosystem services}
% (Chapter~\ref{chap:planetary_boundaries})
% necessary for human health and wellbeing.
% Beyond these basics, some people view the nonhuman world as having inherent 
% worth or standing and, therefore, to be worth preserving.
% For some people, the inclination for preservation is limited to those parts
% of the nonhuman world that humans find appealing, like flowers and songbirds.
% Other people would include even those parts of the nonhuman world that negatively
% affect humanity, like smallpox\index{smallpox}.
% Unfortunately, humans don’t know clearly what pieces of the ecosystem\index{ecosystem}
% are, in the long run, necessary for our survival and which aren’t.
% For instance, could we survive in a world without dandelions?
% Maybe.
% On the other hand, dandelions might be necessary for other organisms we depend on.
% Environmental science views the ecosystem\index{ecosystem} in its entirety as a web, with all parts
% depending on all other parts.
% The ecosystem\index{ecosystem} is not a collection of individual components with binary ``needed''/``not
% needed'' classifications, but as a whole that exists on a continuum from ``fully functional''
% to ``nonfunctional.''
% The choices and paths to sustainability are bristling with value judgments and
% moral choices about what to value and how much.
% \href{https://en.wiktionary.org/wiki/concomitant}{Concomitant} with such choices
% is a weighting of appropriate risks to the ecosystem\index{ecosystem} and human society.
% Different answers to these questions come from different \emph{a priori} assumptions and values.
% 
% The broadest definitions of sustainability\index{sustainability!definition}
% also include the products of human civilization.
% For example, although they are great cultural and historical artifacts,
% humanity could survive without the great pyramids of Giza\index{Giza}.
% On the other hand, we may not survive if we don’t give up coal-fired power plants\index{coal-fired power plant}.
% (And, maybe many people would perish if the coal-fired power plants\index{coal-fired power plant}
% were all suddenly switched off.)
% 
% With respect to Figure \ref{fig:venn_diagram}, environmental sustainability\index{environmental sustainability}
% considers biophysical and thermodynamic constraints and includes issues such as
% pollution\index{pollution}, resource depletion, habitat loss\index{habitat loss}, and biodiversity\index{biodiversity}.
% Harvesting timber\index{timber} faster than it can grow is unsustainable.
% Eventually, deforestation\index{deforestation} means that timber harvesting\index{timber harvesting} must stop because there
% will be no more forest to harvest.
% Depleting \textbf{mineral resources}\index{mineral resources} is unsustainable.
% Eventually, the minerals\index{minerals} will be used up.
% Pumping \textbf{aquifers}\index{aquifer} faster than they can regenerate is unsustainable.
% Eventually, the aquifers will run dry.
% 
% Economic sustainability\index{economic sustainability}
% involves questions of profit and loss, wealth management, and
% macroeconomic policy.
% A business that continually loses money is not sustainable.
% Eventually, it will go out of business.
% 
% Social sustainability\index{social sustainability} comprises human\index{human rights} and 
% civil rights\index{civil rights},
% suffering, and personal freedom.
% A social group that continually loses members, for example, the Whig party, will
% cease to be a group.
% Often, a discussion about social sustainability\index{social sustainability} 
% includes things that should not be sustained but that, typically, have persisted 
% a very long time, like poverty, class inequality, sexism, slavery, and other civil injustices.
% 
% This book discusses issues associated with each of these three
% sustainability domains but with a focus on energy and carbon emissions\index{CO$_2$ emissions}.
% The end-of-chapter questions typically focus on interactions among the three
% domains, like a social-vs.-economic tradeoff.
% 
% %^^^^^^^^^^^^^^^^^^^^^^^^^^^^^^
% \begin{mcframe}[0.90\textwidth](0.85\textwidth)
% %^^^^^^^^^^^^^^^^^^^^^^^^^^^^^^
% This book uses energy and carbon emissions\index{CO$_2$ emissions} as prime examples not
% because they are the only sustainability problems, but because
% energy is the \textbf{master resource}\index{master resource (energy)} and \textbf{climate change}\index{climate change}
% is currently one of the most urgent sustainability problems.
% Energy and carbon emissions\index{CO$_2$ emissions}
% link together many of our sustainability challenges\index{sustainability challenges}.
% \end{mcframe}
% 
% The most oft-quoted~\cite{Quental2011} definition of sustainability\index{sustainability!definition} is that
% \textbf{sustainable development}\index{sustainable development} ``meets the needs of the present without
% compromising the ability of future generations to meet their own needs''~\cite{Brundtland}.
% Note that sustainable development\index{sustainable development} is an oxymoron, %and is not the same as sustainability.
% because ``development'' implies constant improvement, which cannot be sustained
% forever on a planet with finite resources.
% Furthermore, ``needs'' are subjective.
% This definition\index{sustainability!definition} emphasizes the social\index{social sustainability} and
% economic\index{economic sustainability} aspects of sustainability over,
% or instead of, environmental sustainability\index{environmental sustainability}.
% In the end, without environmental sustainability\index{environmental sustainability},
% there is no sustainability at all.
% 
% 
% %%%%%%%%%%%%%%%%%%%%%%%%%%%%%%%%%%%%%%%%%%%%%%%%%%%%%%%%%%%%%%
% \section{A framework for environmental sustainability thinking}\index{environmental sustainability}
% %%%%%%%%%%%%%%%%%%%%%%%%%%%%%%%%%%%%%%%%%%%%%%%%%%%%%%%%%%%%%%
% 
% Two mathematical identities, \textbf{IPAT}\index{IPAT identity}
% and \textbf{Kaya}\index{Kaya identity}, have been used to express the impact
% of human activities on the environment.
% IPAT\index{IPAT identity} expresses ``impact''~\Iparen{} on the environment as the product of human
% \textbf{population}~\Pparen{} \textbf{affluence}~\Aparen{}, and technology~\Tparen{}.
% The Kaya identity is a form of the IPAT identity\index{IPAT identity}
% but restricts environmental impacts~\Iparen{}
% to CO$_2$ emissions\index{CO$_2$ emissions}~(\:\!$\mft{I}_{\text{CO}_2}$).
% On the other hand, Kaya expands the generic expression for ``technology'' to be
% the product of \textbf{primary energy intensity}\index{primary energy intensity}
% of the economy
% (\REp{}, energy per unit of gross domestic product, GDP)
% and the \textbf{carbon intensity of energy}\index{carbon intensity of primary energy}
% (\;\!\XCOtwo{}, CO$_2$ emissions\index{CO$_2$ emissions} per unit of energy).
% 
% Chapters~\ref{chap:planetary_boundaries}--\ref{chap:impact_intensity}
% of this book are organized around a hybrid of the IPAT\index{IPAT identity} and Kaya
% approaches.
% We keep the more general ``impact'' of IPAT\index{IPAT identity} but also generalize the expanded
% ``technology'' expression from Kaya as the product of \textbf{resource intensity of the economy}~\Rparen{}
% and the \textbf{impact of resources}~\Xparen{}.
% 
% \begin{equation} \label{eq:IPARX}
%   \iparx
% \end{equation}
% 
% In the \textbf{IPARX}\index{IPARX identity} formulation, impact~\Iparen{} is a list of impacts (that is, a vector quantity)
% that includes such things as
% \textbf{global warming potential (GWP)}\index{global warming potential (GWP)},
% aquifer depletion\index{aquifer depletion} , and
% eutrophication\index{eutrophication} potential.
% Population~\Pparen{} is the number of people in the world.
% Affluence~\Aparen{} is GDP per capita per year.
% Resource intensity of economic activity~\Rparen{} is the list (vector) of all the resources necessary to
% produce one unit of world GDP.
% Last, impact of resources~\Xparen{} is a list of the impacts of
% each type of resource. (That is, \mft{X} is a matrix quantity.)
% 
% In broad strokes, sustainability can be seen in the \mft{I} term (environmental impacts) and
% in resource extraction\index{resource extraction} (the numerator of \mft{R} and the denominator of \mft{X}\:).
% If we emit wastes\index{waste} at a rate greater than can be assimilated by the environment
% (\:\!\mft{I} too large), we are unsustainable.
% If we withdraw resources from the environment at a rate greater than their regeneration rate,
% we are unsustainable.
% 
% %^^^^^^^^^^^^^^^^^^^^^^^^^^^^^^
% \begin{mcframe}[0.90\textwidth](0.85\textwidth)
% %^^^^^^^^^^^^^^^^^^^^^^^^^^^^^^
% The IPARX identity\index{IPARX identity} is a static relationship.
% Its terms represent steady-state levels,
% but do not necessarily show how changes in any one variable affects the others.
% For instance, using resources more efficiently (improving resource intensity),
% does not, in general, lower impact.
% Instead, it leads to more affluence;
% see Chapter~\ref{chap:resource_intensity}.
% Likewise, it may not be possible to drive the impact(s) per unit of resource to
% zero because of diminishing returns on efficiency
% and tradeoffs that exist between different types of impacts.
% Thus, while Equation~\ref{eq:IPARX} is useful as a conceptual framework for
% thinking about sustainability challenges\index{sustainability challenges}, it does not provide a complete
% roadmap for sustainability solutions (in part) because of interactions among the terms.
% \end{mcframe}
% 
% IPARX\index{IPARX identity} is true because it is an identity, and it provides a useful organizing framework
% for the following chapters.
% To illustrate the IPARX\index{IPARX identity} framework, energy and CO$_2$ are prime examples
% in the rest of the text, because energy is the master resource\index{master resource (energy)}
% and climate change is our most urgent sustainability challenge\index{sustainability challenges!climate change}.
% Of course, the two are closely linked.
% 
% 
% %%%%%%%%%%%%%%%%%%%%%%%%%%%%%%%%%%%%%%%%%%%%%%%%%%%%%%%%%%%%%%
% \section{The mathematics of sustainability}
% \label{sec:exp}
% %%%%%%%%%%%%%%%%%%%%%%%%%%%%%%%%%%%%%%%%%%%%%%%%%%%%%%%%%%%%%%
% 
% Returning to the king's wager\index{king's wager}, for a series of doubling numbers,
% like the number of grains on the scholar's chessboard\index{chessboard} squares:
% 1, 2, 4, 8, 16, 32, \ldots,
% the sum of all the
% numbers in the series is always one less than the next number in the series.
% For example,
% 1+2+4+8 $=$ 15, which is one less than 16, and
% 1+2+4+8+16 $=$ 31, which is one less than 32.
% Thus, the amount of grain on the next square is always greater than the sum total
% of all the grain on all previous squares.
% The total grain on \mft{N} squares is 2$^{\mft{N}}-$~\text{1}.
% The amount of grain on the \mft{N}$^\text{th}$ square is 2$^{\mft{N} - \text{1}}$,
% which is half of 2$^\mft{N}$
% and one more than the sum of all previous $\mft{N} - \text{1}$ squares.
% 
% Doubling on discrete squares (or in discrete time units) can be extended to
% continuous time.
% Exponential growth\index{exponential growth} is growth at a constant rate (in percent per time),
% which means doubling in a fixed amount of time.
% That is, the time it takes to double does not depend on the current amount.
% If the (constant) rate of growth is \mft{r}, then amount of time to double is ln(2)/\mft{r},
% which is approximately 70 divided by the growth rate.
% For example, with a fixed 3\%/yr growth rate, GDP doubles every 23 years.
% If your investment portfolio has a 7\%/yr return, your wealth will double
% every 10 years.
% 
% 
% <<exp, fig.scap="Mass of grain on the chessboard\\index{chessboard} in the king's wager\\index{king's wager}", fig.cap="Mass of grain on each square of the chessboard in the king's wager. The last row of the chessboard (squares 57--64) contains nearly all of the grain. Pg is petagram; 1~Pg $=$ 10$^{\\text{12}}$~kg.">>=
% chessboard_data <- tibble::tibble(x = 1:64) %>%
%   dplyr::mutate(
%     y = 2^(x - 1),
%     m = 0.065 / 1e15 * y   # 65 mg/grain, mass in Pg.
%   )
% 
% chessboard_data %>%
%   ggplot2::ggplot(mapping = ggplot2::aes(x = x, y = m)) +
%   ggplot2::geom_hline(yintercept = 0, colour = guide_line_colour, size = guide_line_size, linetype = guide_linetype) +
%   ggplot2::geom_bar(stat = "identity", fill = bar_colour) +
% 
%   ggplot2::scale_x_continuous(limits = c(0, 65),
%                               breaks = c(1, 8, 16, 24, 32, 40, 48, 56, 64)) +
% 
%   ggplot2::labs(x = "Chessboard square",
%                 y = "Rice mass [Pg]") +
% 
%   MKHthemes::xy_theme()
% @
% 
% Figure~\ref{fig:exp} shows the mass of grain on each square of the chessboard\index{chessboard}.
% The sharp increase at the right side of the figure
% is characteristic of exponential growth\index{exponential growth} over any time scale.
% If the chessboard\index{chessboard} were extended by another row, the new row (squares 65--72)
% would make current last row (squares 57--64) look as small and insignificant as
% squares 49--56 currently appear.
% 
% Figure~\ref{fig:exp} can represent other quantities that are growing exponentially\index{exponential growth},
% like GDP.
% The blue area then represents the resources needed to support that activity.
% For instance, the blue area could represent the amount of energy necessary to create
% that amount of GDP.
% (The various resources needed by the economy to produce a unit of GDP may be assumed
% to be constant over a short period of time but are not constant long term;
% see Chapter~\ref{chap:resource_intensity}.)
% 
% Population~\Pparen{} and affluence~\Aparen{}
% in Equation~\ref{eq:IPARX} have been growing exponentially\index{exponential growth}.
% Chapter~\ref{chap:population} gives us some hope that population won't continue
% growing exponentially\index{exponential growth}.
% (In fact, there are some indications world population may fall after a peak
% in the mid-21$^{\mathrm{st}}$ century.)
% However, world economies are actively managed to have continual growth
% (perhaps, an average target of 3\%/yr).
% The implication for sustainability is that the resources---for instance,
% energy---needed by the economy in the next 23 years will be more than the total
% resources (energy) used in all of history to date.
% Because of this exponential economic growth\index{exponential growth}, resource use grows continually to infinity.
% Humanity is not on a sustainable path.
% 
% 
% %%%%%%%%%%%%%%%%%%%%%%%%%%%%%%%%%%%%%%%%%%%%%%%%%%%%%%%%%%%%%%
% \section{Energy is the master resource}\index{master resource (energy)}
% \label{sec:masterE}
% %%%%%%%%%%%%%%%%%%%%%%%%%%%%%%%%%%%%%%%%%%%%%%%%%%%%%%%%%%%%%%
% 
% At a very fundamental level,
% betterment of the human condition occurs when ingenuity is applied
% in economies to transform materials into useful products.
% Thus, both mass (for the materials) and energy (for the transformation)
% are crucial for improvement of human life.
% Both mass and energy come from the environment.
% 
% Although a wide variety of materials are essential for human existence
% (Section~\ref{sec:resources_strong_sustainability}),
% we say that energy is the master resource\index{master resource (energy)}, because
% obtaining and using any other resource
% (including extraction\index{resource extraction} of materials from the environment)
% requires energy.
% Collecting firewood\index{firewood} to heat\index{wood heating} houses requires energy.
% Extracting crude oil requires energy.
% Harvesting food requires energy.
% Making solar panels\index{solar panel} requires energy.
% 
% Another reason we say energy is the master resource\index{master resource (energy)}
% is that energy can be used to convert one resource into another.
% Crude oil\index{crude oil} can be refined to gasoline\index{gasoline}.
% Corn\index{corn} can be converted to ethanol\index{corn ethanol}.
% Salt water can be converted to fresh water.
% Petroleum\index{petroleum} or natural gas\index{natural gas} can be converted to pharmaceuticals\index{pharmaceuticals}
% or plastics.
% 
% Finally, energy is the master resource\index{master resource (energy)} because
% it is needed to break down and dispose of wastes\index{waste}.
% Building a landfill takes energy.
% Hauling garbage to the landfill takes energy.
% Even the decomposition of compost\index{compost} takes sunlight, air, and rain.
% 
% Because energy is vitally important,
% metrics have been developed to describe its use.
% For example, \textbf{energy efficiency}\index{energy efficiency} can be calculated
% for the use of energy in any process.
% Energy efficiency\index{energy efficiency}~($\eta$) is defined as
% 
% \begin{equation} \label{eq:energy_efficiency_def}
%   \eta \equiv \frac{\text{Energy out}}{\text{Energy in}} \; .
% \end{equation}
% 
% A different metric is often calculated for the production of energy:
% \textbf{energy return on investment (EROI)}\index{energy return on investment (EROI)},
% which is defined as the amount of energy delivered to the economy
% per unit of energy it took to deliver that energy,
% not including feedstock\index{feedstock}%
% \footnote{
%   ``Not including feedstock\index{feedstock}'' means that only auxiliary energy inputs should be counted
%   in the denominator.
%   For example, when calculating the EROI\index{energy return on investment (EROI)} for an oil rig,
%   only the electricity\index{electricity} and diesel\index{diesel} to operate the rig is counted in the denominator.
%   The crude oil\index{crude oil} extracted by the rig is not counted in the denominator.
% }:
% 
% \begin{equation}\label{eroi}
%    \text{EROI} \equiv \frac{\text{Useful energy delivered to society}}
%                            {\text{Energy required to obtain delivered energy}} \; .
% \end{equation}
% %
% Because energy is the master resource\index{master resource (energy)},
% understanding EROI\index{energy return on investment (EROI)} is vitally important.
% An EROI\index{energy return on investment (EROI)}
% that is less than one indicates the energy production activity is an energy sink\index{energy sink} for society.
% For instance, modern agriculture has an EROI\index{energy return on investment (EROI)} of around 0.1 to 0.15;
% it takes 7--10 calories of energy to deliver 1 calorie of food energy.
% On the other hand, an EROI\index{energy return on investment (EROI)} greater than one indicates 
% an energy source\index{energy source} for society.
% Table~\ref{tab:eroi} lists energy returns\index{energy return on investment (EROI)}
% for various sources of energy.
% Some authors have suggested a minimum EROI\index{energy return on investment (EROI)}
% of~5 is required for sustainability~\cite{Ferruccio2016}
% with 12--13 necessary for a technological society.
% Although constrained by basic biophysical limits, 
% EROI\index{energy return on investment (EROI)} can be improved with better,
% more efficient technology.
% 
% \begin{table}\centering
% \caption[EROI for several energy sources]
%         {Energy return on investment (EROI)\index{energy return on investment (EROI)}
%         for several energy sources.
% \emph{Sources}:~\citet{Wikipedia_EROI}, \citet{WNA2020}, \citet{EtOH_in_BR}, \citet{Gomiero2015}, and sources therein.}
% \label{tab:eroi}
% \begin{tabular}{ll}
% \toprule
% Energy source                                 & EROI\index{energy return on investment (EROI)}
% \\ \midrule
% Nuclear\index{nuclear energy}                 & 59--106 \\
% Hydroelectric\index{hydro electricity}        & 45+ \\
% Coal\index{coal} and natural gas\index{natural gas} & 28--31 \\
% Concentrated solar thermal                    & 21 \\
% Wind energy                                   & 15--35+ \\
% Brazilian sugar-cane ethanol\index{sugar-cane ethanol}         & 8.6--10.2\\  % https://en.wikipedia.org/wiki/Ethanol_fuel_in_Brazil
% Solar photovoltaic\index{solar panel}         & 2--8 \\
% Biofuels\index{biofuel} (incl.\ U.S.\ corn ethanol\index{corn ethanol})  & 0.8--1.6 \\ %%  from T. Gomiero 2015
% \bottomrule
% \end{tabular}
% \end{table}
% 
% %%%%%%%%%%%%%%%%%%%%%%%%%%%%%%%%%%%%%%%%%%%%%%%%%%%%%%%%%%%%%%
% \section{Summary}
% %%%%%%%%%%%%%%%%%%%%%%%%%%%%%%%%%%%%%%%%%%%%%%%%%%%%%%%%%%%%%%
% 
% This book uses the IPARX identity\index{IPARX identity} as a framework for 
% thinking about sustainability.
% It considers many different resources and many different types of impacts.
% However, there is a focus on energy and CO$_2$ emissions\index{CO$_2$ emissions}.
% Climate change\index{climate change} (Section \ref{sec:GHGCC}) is one of the most
% urgent sustainability challenges, and it is linked to other urgent challenges 
% (such as material extraction rates and land-use changes) by CO$_2$ and energy.
% Thus, carbon emissions\index{CO$_2$ emissions} and energy lie at the nexus of 
% our sustainability challenges\index{sustainability challenges}.
% 



\end{document}

