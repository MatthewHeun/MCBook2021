
\def\HALFTITLE{A Framework for Sustainability\\ Thinking: \\%The Multifaceted Nature\\
		%	of Sustainability Challenges:\\
			An Engineering Perspective \\ 
			\textcolor{red}{version $\alpha$ \\}
			\textcolor{black}{\today}}
\def\TITLE{A Framework for Sustainability Thinking \\ %The Multifaceted Nature of Sustainability Challenges\\ 
                        \textcolor{red}{version $\alpha$} \\ \textcolor{black}{\today}}
\def\AUTHORA{Jeremy Van Antwerp}
\def\AFFILIATIONA{Engineering Department, Calvin University}
\def\AUTHORB{Matthew Kuperus Heun}
\def\AFFILIATIONB{Engineering Department, Calvin University}
\def\AUTHORS{\AUTHORA\ and \AUTHORB}
\def\LECTURE{\ \#4}
\def\lcSYNTHESIS{Synthesis Lectures on Sustainable Development}
\def\SYNTHESIS{\MakeUppercase{\textit{\lcSYNTHESIS}}}

\def\EDITOR{xxxx, \textit{yyyy}}


%%%%%%%%%%%% HALF TITLE PAGE 1-2

\thispagestyle{emptyrule}

\halftitle{\HALFTITLE}

\clearpage


%%%%%%%%%%%% FULL TITLE 5

\blankpage

\thispagestyle{emptyrule}
\title{\TITLE}

\vspace*{2pc}
\authorname{\AUTHORA}
\authoraffiliation{\AFFILIATIONA}

\vspace*{1pc}
\authorname{\AUTHORB}
\authoraffiliation{\AFFILIATIONB}

\vfill
\synthesis{\SYNTHESIS\LECTURE}
\morganlogo

\clearpage


%%%%%%%%%%%% ABSTRACT AND KEYWORDS   6

\thispagestyle{emptyrule}

\ABSTRACT
\noindent
%The abstract goes here. 
%The Abstract and the keywords have to fit in this page.
This book is intended to be an introduction to the many challenges of 
sustainability.
It provides basic facts, figures, and information related to sustainability
in a way that is easy to digest.
The focus is on a quantitative understanding of the challenges of sustainability.
The first half of the book develops a framework for sustainability thinking.
The second half considers application areas and personal and corporate responses
to sustainability challenges.
Throughout, the end-of-chapter discussion questions focus on tradeoffs among 
competing goods and the ethical and social implications of decisions related to
sustainability.
This book was written for a university seminar course on sustainability but could
be used in other small-group discussion formats. 
It is intended to be easy to read but hard to digest.
%end ABSTRACT

\keywords{% %% take a look in the index to see what should get added as keywords!
%xxx, yyyy, zz
Sustainability, sustainable development, systems modeling
}

\vfill

\clearpage


%%%%%%%%%%%%% DEDICATION  7-8


%{
%\clearpage
%\thispagestyle{plain}
%
%\vspace*{13pc}\Large\it
%\centerline{To Eric, Jacob, and my parents.}
%
%}
%
%\clearpage
 

%%%%%%%%%%%%%%%%%%%%%%%% TOC %%%%%%%%%%%%%%%%%%

%blankpage

{
\pagestyle{plain}
\tableofcontents
}

\clearpage
 
 
%%%%%%%%%%%%%%%%%%%%%%%% PREFACE %%%%%%%%%%%%%%%%%%%
 
 
\blankpage

{
\chapter*{Preface}
\addcontentsline{toc}{chapter}{\protect\numberline{}{Preface}}
\thispagestyle{plain}
\markboth{PREFACE}{PREFACE}

\noindent
\section*{Who should use this book} Anyone interested in sustainability. 
This book is intended to be an introduction to the many challenges of 
sustainability and, as such, does not assume any prior background knowledge,
apart from just a little bit of math.
The book is not intended to be comprehensive but is intended to be easy to read.\\

\section*{How to use this book} Read it. Seriously; that's what you do with books.
This book was written to support a one semester-hour university seminar class
on sustainability. The primary focus of class time is in-class discussion, so 
the end-of-chapter discussion questions should provide significant value. 
Therefore, individual chapters, or sets of chapters, can easily be
used to support a wide variety of class structures, types, and levels.
Students should read the text on their own, out of class, to get 
basic information related to  the challenges of sustainability. We encourage 
instructors to use a basic and straightforward quiz on the reading through your 
course management platform to hold students accountable for the reading. 
Students should review discussion questions ahead of each class and 
prepare for discussion. \\

\section*{What you should get out of this book} This book contains four main 
contributions.
\begin{itemize}
\item Basic facts, figures, and information related to sustainability. 
We don't assume you start with any prior knowledge but by the time you finish 
reading this book you \emph{will} know basic facts like carbon dioxide 
concentration in the atmosphere and what are the anthropogenic sources of 
carbon dioxide. 
Furthermore, you should have a \emph{sense of scale} related to sustainability 
information. 
That is, which things are big and important and which are smaller and less relavant. 
To this end, the information is presented \emph{graphically} as much as possible 
to show relative magnitudes.
\item We try to provide an explanatory framework or scaffold thinking about sustainability,
primarily in chapters \ref{chap:introduction}-\ref{chap:carbon_intensity}.
\item The end-of-chapter discussion questions get at moral, ethical, philosophical,
and practical aspects of sustainability. They don't have %\sout{right} 
objective answers. 
Instead, the questions should help illustrate the roll that worldview, values, 
preferences, and \emph{a priori} assumptions have in how we approach sustainability. 
%social pillar?
\item The end-of-chapter project questions range in difficulty from a long
homework problem to a graduate thesis project. If graduate study on a sustainability
topic is in your future, perhaps you can take inspiration from one or more of 
these project questions. Instructors can use one -- or perhaps more -- of these 
questions for a semester-long project, optionally with in-class presentations, for a class.
\end{itemize}
By the end of the book you should see that sustainability is an important, 
urgent, difficult, but solvable problem that requires personal and corporate 
commitments. \\

\section*{Text, organization, and schedule} At one chapter a week, the twelve chapters
in this book don't quite fill a typical university semester. Of course, you don't
have to be in a university context to use this book. A book club or church small group 
are also appropriate audiences -- sustainability is important for everyone. However,
if you are in a university context, here are a few ways that you could fill the 
remaining weeks of your semester. Use class time for project presentations, as
mentioned above. Perhaps groups of students could each be assigned different 
project questions and at the end of the semester, each could share their results.
Alternatively, a class session or sessions could be devoted to a topic or topics your instructor
finds personally interesting, like biofuels, grid-scale storage, or media coverage
of sustainability. At Calvin University, our class devotes a week to worldview
implications for sustainability. You could use our paper \emph{Current disciplines 
and worldviews are insufficient to address sustainability challenges} (2019) as a 
starting place for discussion. % CES website doesn't have 2019 proceedings!
On the other hand, instructors don't need to use all of the book. IEEE Xplore
makes it easy to assign either a single chapter or a group of chapters as supplementary
material for many different types of classes.
Lastly, if you read all the way to the end of the Preface, great job! Give youself a 
gold star. Now, on to the good stuff...


\vspace*{2pc}
\noindent\AUTHORS\\
\noindent December 2021
}

\clearpage


%%%%%%%%%%%%%%%%%%%%%%%% ACK %%%%%%%%%%%%%%%%%%%


\blankpage

\chapter*{Acknowledgments}
\addcontentsline{toc}{chapter}{\protect\numberline{}{Acknowledgments}}
\thispagestyle{plain}
\markboth{ACKNOWLEDGMENTS}{ACKNOWLEDGMENTS}

\noindent
Thanks to the students of the Sustainability Challenges course and our colleague 
Julie Wildschut for teaching the class in fall 2021.

Thanks to research assistants Larisa Tomeci, Henos Tadesse, and Joshua Broekhuisen
for help with sourcing data and creating figures for the text. 
% was there a 4th student?

Thank you to the many reviewers who provided feedback on early versions to the book:
James VanAntwerp, Julie Wildschut, Becky Haney, James Skillen, Uko Zylstra.

% Jeff Jewett, Shannon Savage, Leonides Murembya, Steve McMullen, Nathan Grawe
% Jennifer VanAntwerp,  Glen VanAntwerp, ...
% Derek Schuurman, Sa BA, Tom Ackerman

Thanks to the Calvin University Board of Trustees and Joel and Linda Zylstra
for funding to create and revise the text.
Thanks also to Tim Koning of Liquid Haulers Maintenance for funding on biofuels research.



\vspace*{2pc}
\noindent\AUTHORS\\
\noindent December 2021
 
\clearpage

\blankpage

