% The next command tells RStudio to do "Compile PDF" on book.Rnw,
% instead of this chapter, thereby eliminating the need to switch back to book.Rnw 
% before making the book.
%!TEX root = book.Rnw

\def\HALFTITLE{A Framework for \\ 
                Sustainability Thinking: \\%The Multifaceted Nature\\
		%	of Sustainability Challenges:\\
			A student's introduction to global sustainability challenges \\ 
			%\textcolor{red}{version $\beta$ \\}
			%\textcolor{black}{\today}
			}
\def\TITLE{A Framework for Sustainability Thinking } %\\ %The Multifaceted Nature of Sustainability Challenges\\ 
                      %  \textcolor{red}{version $\beta$} \\ \textcolor{black}{\today}}
\def\AUTHORA{Jeremy Van Antwerp}
\def\AFFILIATIONA{Engineering Department, Calvin University}
\def\AUTHORB{Matthew Kuperus Heun}
\def\AFFILIATIONB{Engineering Department, Calvin University}
\def\AUTHORS{\AUTHORA\ and \AUTHORB}
\def\LECTURE{\ \#4}
\def\lcSYNTHESIS{Synthesis Lectures on Sustainable Development}
\def\SYNTHESIS{\MakeUppercase{\textit{\lcSYNTHESIS}}}

\def\EDITOR{xxxx, \textit{yyyy}}


%%%%%%%%%%%% HALF TITLE PAGE 1-2

\thispagestyle{emptyrule}

\halftitle{\HALFTITLE}

\clearpage


%%%%%%%%%%%% FULL TITLE 5

\blankpage

\thispagestyle{emptyrule}
\title{\TITLE}

\vspace*{2pc}
\authorname{\AUTHORA}
\authoraffiliation{\AFFILIATIONA}

\vspace*{1pc}
\authorname{\AUTHORB}
\authoraffiliation{\AFFILIATIONB}

\vfill
\synthesis{\SYNTHESIS\LECTURE}
\morganlogo

\cleardoublepage


%%%%%%%%%%%% ABSTRACT AND KEYWORDS   6

\thispagestyle{emptyrule}

\ABSTRACT
\noindent
%The abstract goes here. 
%The Abstract and the keywords have to fit in this page.
This book is an introduction to the many challenges of sustainability.
The first half of the book develops a framework for sustainability thinking.
The second half considers application areas and personal and corporate responses
to sustainability challenges.
Basic facts, figures, and information related to sustainability are presented
in a way that should convey to readers a sense of scale for the many sustainability challenges
we face.
%The objective is for readers to develop a sense of scale related to sustainability challenges.
Throughout, the end-of-chapter projects and discussion questions focus on tradeoffs among 
competing goods and the ethical and social implications of decisions related to
sustainability.
This book was written for a university seminar course on sustainability but could
be used in other small-group discussion formats. 
It is intended to be easy to read but hard to digest.
%end ABSTRACT

\keywords{% %% take a look in the index to see what should get added as keywords!
%xxx, yyyy, zz
Sustainability, framework, climate change, sustainability discussion.
}

\vfill

\cleardoublepage


%%%%%%%%%%%%% DEDICATION  7-8


{
\clearpage
\thispagestyle{plain}

\vspace*{13pc}\Large\it
\centerline{To our past, present, and future students.}
}

\cleardoublepage
 

%%%%%%%%%%%%%%%%%%%%%%%% TOC %%%%%%%%%%%%%%%%%%

%blankpage

{
\pagestyle{plain}
\tableofcontents
}

\cleardoublepage
 

%%%%%%%%%%%%%%%%%%%%%%%% List of figures %%%%%%%%%%%%%%%%%%

%blankpage

{
\pagestyle{plain}
\listoffigures
}

\cleardoublepage
 

%%%%%%%%%%%%%%%%%%%%%%%% List of tables %%%%%%%%%%%%%%%%%%

%blankpage

{
\pagestyle{plain}
\listoftables
}

\cleardoublepage
 
 
%%%%%%%%%%%%%%%%%%%%%%%% PREFACE %%%%%%%%%%%%%%%%%%%
 
 
% \blankpage

{
\chapter*{Preface}
\addcontentsline{toc}{chapter}{\protect\numberline{}{Preface}}
\thispagestyle{plain}
\markboth{PREFACE}{PREFACE}

\noindent


%%%%%%%%%%%%%%%%%%%%%%%%%%%%%%%%%%%%%%%%%%%%%%%%%%%%%%%%%%%%%%
\section*{Who should use this book} 
%%%%%%%%%%%%%%%%%%%%%%%%%%%%%%%%%%%%%%%%%%%%%%%%%%%%%%%%%%%%%%

This book is for anyone who is interested in sustainability.
It introduces the many challenges of sustainability
and it provides a framework for thinking about sustainability.
The reader is not assumed to have prior background knowledge,
apart from basic numeracy and the ability to comprehend graphical information.
This book is not intended to be comprehensive in its coverage of all aspects of 
sustainability.
Neither does it suggest solutions to many sustainability problems. 
It should be easy to read but hard to digest.


%%%%%%%%%%%%%%%%%%%%%%%%%%%%%%%%%%%%%%%%%%%%%%%%%%%%%%%%%%%%%%
\section*{How to use this book} 
%%%%%%%%%%%%%%%%%%%%%%%%%%%%%%%%%%%%%%%%%%%%%%%%%%%%%%%%%%%%%%

This book was written to support a university seminar class %one semester-hour 
on sustainability. 
The primary focus of class time is in-class discussion, so 
the end-of-chapter discussion questions should provide significant value. 
Therefore, this book could be used in other settings that prioritorize discussion,
like a book club or small group.
Individual chapters, or sets of chapters, can easily be
used to support a wide variety of class structures, types, and levels.
Students should read the text on their own, out of class, to get 
basic information related to  the challenges of sustainability. 
We encourage instructors to use a basic and straightforward quiz on the reading 
through your course management platform to hold students accountable for the 
reading. 
Students should review discussion questions ahead of each class and 
prepare for discussion.


%%%%%%%%%%%%%%%%%%%%%%%%%%%%%%%%%%%%%%%%%%%%%%%%%%%%%%%%%%%%%%
\section*{What you should put into this book} 
%%%%%%%%%%%%%%%%%%%%%%%%%%%%%%%%%%%%%%%%%%%%%%%%%%%%%%%%%%%%%%

It is often said that ``you only get out what you put in'' to an activity.
We encourage readers to be critical.
What other sustainability concerns do you have?
What other ideas can you suggest to become more sustainable?

As you read, you will do well to keep the table of key terms shown below at hand.

\begin{table}[h!]
% No caption, because it appears as Table 0.1, 
% which is silly.
\caption{Key terms.}
\centering
\begin{tabular}{r l}
\toprule
Key term              & Meaning      \\ 
\midrule
The economy           & Society's metabolism. \vspace{1mm} \\
%
\multirow{2}{*}{Efficacy}   & The ratio $\frac{\text{useful output}}{\text{input}}$, with different units \\
                            & in numerator and denominator. \vspace{1mm} \\
%
Efficiency            & A unitless ratio $\frac{\text{useful output}}{\text{input}}$, often for energy. \vspace{1mm} \\
%
\multirow{3}{*}{EROI} & Energy return on investment, a key metric for \\
                      & energy production systems, defined as $\frac{\text{energy produced}}{\text{energy input}}$, \\
                      & where ``energy input'' does not include feedstock. \vspace{1mm} \\
%
\multirow{3}{*}{Energy service}   & The purpose of energy consumption, \\
                                  & for example transportation, illumination, \\
                                  & or space heating. \vspace{1mm} \\
%
\multirow{2}{*}{Externality}   & A cost or benefit borne by a third party \\ 
                               & to an economic transaction. \vspace{1mm} \\
%
\multirow{2}{*}{Intensity}  & The ratio $\frac{\text{input}}{\text{output}}$, for example
                              $\mft{R} = \frac{\text{resource consumption}}{\text{GDP}}$, \\
                            & with different units in numerator and denominator. \vspace{1mm} \\
%
\multirow{3}{*}{Productivity} & The ratio $\frac{\text{economic output (\$)}}{\text{production input}}$, a type of efficacy, \\
                              & often with different units in numerator~(\$) and  \\
                              & denominator (capital, labor, energy, or materials). \vspace{1mm} \\
%
\multirow{2}{*}{Resources}  & Material extracted from the natural environment \\
                            & for purposes of human consumption. \vspace{1mm} \\
%
\multirow{2}{*}{Strong sustainability} & Using resources more slowly than \\
                                       & natural regeneration rates. \vspace{1mm} \\
%
Weak sustainability         & Using resources so they last 50~years. \\
\bottomrule
\end{tabular}
\label{tab:key_terms}
\end{table}



%%%%%%%%%%%%%%%%%%%%%%%%%%%%%%%%%%%%%%%%%%%%%%%%%%%%%%%%%%%%%%
\section*{What you should get out of this book} 
%%%%%%%%%%%%%%%%%%%%%%%%%%%%%%%%%%%%%%%%%%%%%%%%%%%%%%%%%%%%%%t

This book contains four main contributions.

\begin{itemize}

  \item Basic facts, figures, and information related to sustainability. 
        We don't assume you start with any prior knowledge, but by the time you finish 
        reading this book you will know basic facts like carbon dioxide 
        concentration in the atmosphere. 
        Furthermore, you should gain a sense of scale related to sustainability 
        information. 
        That is, which things are big and important and 
        which are smaller and less relavant. 
        To this end, information is presented graphically as much as possible 
        to show relative magnitudes and trends.
        
  \item An explanatory framework for thinking about sustainability,
        primarily in chapters \ref{chap:introduction}--\ref{chap:impact_intensity}.

  \item A wealth of end-of-chapter discussion questions that 
        address moral, ethical, philosophical,
        and practical aspects of sustainability. 
        Most questions don't have objective answers. 
        Instead, they illustrate the importance
        of worldview, values, preferences, and \emph{a priori} assumptions 
        for each person's approach to sustainability. 
        
  \item Inspiration for future sustainability-related studies.
        The end-of-chapter projects range in scope from a long
        homework problem to a graduate thesis. 
        Instructors can use one---or perhaps more---of the projects for
        semester-long or capstone assignmets,
        with optional in-class presentations.
      
\end{itemize}

When you finish Chapter~\ref{chap:personal_action}, you should see
that sustainability entails important, urgent, and difficult challenges; 
that sustainability transitions are needed; and 
that you can play a role in helping the world become more sustainable
through a variety of individual and corporate actions.


%%%%%%%%%%%%%%%%%%%%%%%%%%%%%%%%%%%%%%%%%%%%%%%%%%%%%%%%%%%%%%
\section*{Text, organization, and schedule} 
%%%%%%%%%%%%%%%%%%%%%%%%%%%%%%%%%%%%%%%%%%%%%%%%%%%%%%%%%%%%%%t

At one chapter a week, the twelve chapters in this book don't quite fill a typical 
university semester. 
Of course, you don't have to be in a university context to use this book. 
A book club or small group are also appropriate audiences---sustainability 
is important for everyone. 
However, if you are in a university context, here are a few ways that you could fill the 
remaining weeks of your semester. 
Use class time for project presentations. 
Perhaps groups of students could each be assigned different project questions and, 
at the end of the semester, each could share their results.
Alternatively, one or more class sessions could be devoted to a topic or topics 
of the instructor's choice, like biofuels, grid-scale storage, 
or media coverage of sustainability. 
At Calvin University, our class devotes a week to worldview implications for 
sustainability. 
You could use all or part(s) of \emph{Beyond stewardship: New approaches to 
creation care}~\citep{Warners:2019aa}
or \emph{Current disciplines and worldviews are insufficient
to address sustainability challenges}~\cite{VanH2019} as a starting point for discussion. 
On the other hand, instructors don't need to use all of the book. 
IEEE Xplore makes it easy to assign either a single chapter or a group of chapters 
as supplementary material for many different types of classes.
% Lastly, if you read all the way to the end of the Preface, great job! 
% Give youself a gold star. 
Now, on to the good stuff!

\vspace*{2pc}
\noindent\AUTHORS\\
\noindent January 2022
}

\cleardoublepage


%%%%%%%%%%%%%%%%%%%%%%%% ACK %%%%%%%%%%%%%%%%%%%


%#############################################################
\chapter*{Acknowledgments}
%#############################################################

\addcontentsline{toc}{chapter}{\protect\numberline{}{Acknowledgments}}
\thispagestyle{plain}
\markboth{ACKNOWLEDGMENTS}{ACKNOWLEDGMENTS}

\noindent
Thank you to the many students who have taken the Sustainability Challenges course 
over the years and engaged deeply with course content.
Thank you to our colleague Julie Wildschut for teaching Sustainability Challenges in Fall 2021 so
that we could write this book.

Thank you to student researchers Joshua Broekhuisen, Anjana Sainju, 
Henos Tadesse, and Larisa Tomeci
for help with sourcing data for the figures in the text. 

Serveral reviewers provided feedback on early versions to the book---thank you!
James VanAntwerp provided feedback on an early version of Part~\ref{part:IPARX}.
Margaret VanAntwerp provided extensive feedback as a student in the
Sustainability Challenges course in Fall 2021.
% Other reviewers were Julie Wildschut and Glen VanAntwerp. Becky Haney, James Skillen, Uko Zylstra.
Other reviewers were Paul Brockway, Nathan Grawe, Zeke Marshall,
Gregor Semieniuk, T\^{a}nia Sousa, and Uko Zylstra.

% Jeff Jewett, Shannon Savage, Leonides Murembya, Steve McMullen, Nathan Grawe
% Jennifer VanAntwerp,  Glen VanAntwerp, ...
% Derek Schuurman, Sa BA, Tom Ackerman

Thank you to the Calvin University Board of Trustees and 
to Joel and Linda Zylstra
for funding to create and revise the text.
Thanks also to Tim Koning of LHM Tank for providing funding for biofuels research.



\vspace*{2pc}
\noindent\AUTHORS\\
\noindent January 2022
 
\clearpage

\blankpage
